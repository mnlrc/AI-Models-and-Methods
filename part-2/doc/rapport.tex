\documentclass[12pt]{article}
\usepackage[utf8]{inputenc}
\usepackage{eso-pic,graphicx}
\usepackage[a4paper,left=2cm,right=2cm,top=2cm,bottom=2cm]{geometry}
\usepackage{titlesec}
\usepackage{hyperref}
\usepackage{float}
\usepackage[pages=some]{background}

\newcommand{\HRule}{\rule{\linewidth}{0.3mm}}


% -----------------------------------------------------

\begin{document}

\AddToShipoutPictureBG*{\includegraphics[width=\paperwidth,height=\paperheight]{images/background.png}}
\clearpage
\begin{titlepage}
  \begin{sffamily}
  \begin{flushleft} \large
    \includegraphics[height=2.0cm]{images/logo_ulb.jpg}
    \vspace{5cm}
   \end{flushleft}
  \begin{center}

    %Title
        \textsc{\huge INFO-F311 - Projet d'IA 2}\\[1cm]

    \HRule \\[0.7cm]

        \textsc {\Huge Réseaux Bayésiens}\\[0.4cm]

    \HRule \\[1.2cm]

% Author and supervisor
\begin{minipage}{0.5\textwidth}
\begin{flushleft} \large
\emph{Auteur:}\\
Manuel \textsc{Rocca} - 000596086\\


\end{flushleft}
\end{minipage}
\vspace{1cm}

\begin{minipage}{0.4\textwidth}
\begin{flushright} \large
\emph{Professeurs:} \\
Tom  \textsc{Lenaerts}\\
\emph{Assistants:} \\
Axel \textsc{Abels} \\
Martin \textsc{Colot} \\
Yannick \textsc{Molinghen} \\
Pascal \textsc{Tribel}
\end{flushright}
\end{minipage}


    \vfill

    %Bottom of the page
    {\large Année académique 2025-2026}
  \end{center}

  \end{sffamily}
\end{titlepage}


\clearpage


\tableofcontents
\newpage

% -----------------------------------------------------

\section{Introduction}

Ce second projet d'Intelligence Artificielle a pour but d'appliquer la théorie concernant les réseaux Bayésiens. Cette structure 
de données repose fortement sur les probabilités, en particulier les probabilités conjointes. Donc, en plus de "bêtement" implémenter
une structure de données, ce projet nous a permis de relier différents concepts provenants de divers cours du bachelier de sciences-
informatiques (en particulier Probabilités et Statistiques, cours de BA2).

% ajouter définition + vulgarisation

\section{Implémentation des expériences}

\subsection{Fonction de vraisemblance}
% donner la définition d'un hyperparamètre pour argumenter et être formel

\subsection{Inférence par énumération}
% expliquer comment elle a été implémentée; grâce à romain mais tout dans la subtilité

\section{Analyses de l'impact de paramètres sur la détection de gemmes}

\subsection{Impact du paramètre 𝜆}

\subsection{Impact de l'intensité du bruit $𝜎$}


\section{Usage des LLMs}

\section{Conclusion}



\end{document}
