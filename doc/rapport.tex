\documentclass[12pt]{article}
\usepackage[utf8]{inputenc}
\usepackage{eso-pic,graphicx}
\usepackage[a4paper,left=2cm,right=2cm,top=2cm,bottom=2cm]{geometry}
\usepackage{titlesec}
\usepackage{hyperref}
\usepackage[pages=some]{background}

\newcommand{\HRule}{\rule{\linewidth}{0.3mm}}


% -----------------------------------------------------

\begin{document}

\AddToShipoutPictureBG*{\includegraphics[width=\paperwidth,height=\paperheight]{images/background.png}}
\clearpage
\begin{titlepage}
  \begin{sffamily}
  \begin{flushleft} \large
    \includegraphics[height=2.0cm]{images/logo_ulb.jpg}
    \vspace{5cm}
   \end{flushleft}
  \begin{center}

    %Title
        \textsc{\huge INFO-F311 - Projet d'IA 1}\\[1cm]

    \HRule \\[0.7cm]

        \textsc {\Huge Recherche}\\[0.4cm]

    \HRule \\[1.2cm]

% Author and supervisor
\begin{minipage}{0.5\textwidth}
\begin{flushleft} \large
\emph{Auteur:}\\
Manuel \textsc{Rocca} - 000596086\\


\end{flushleft}
\end{minipage}
\vspace{1cm}

\begin{minipage}{0.4\textwidth}
\begin{flushright} \large
\emph{Professeurs:} \\
Tom  \textsc{Lenaerts}\\
\emph{Assistants:} \\
Axel \textsc{Abels} \\
Martin \textsc{Colot} \\
Yannick \textsc{Molinghen} \\
Pascal \textsc{Tribel}
\end{flushright}
\end{minipage}


    \vfill

    %Bottom of the page
    {\large Année académique 2025-2026}
  \end{center}

  \end{sffamily}
\end{titlepage}


\clearpage


\tableofcontents
\newpage

% -----------------------------------------------------

\section{Introduction}

Durant cette année, au cours d'Intelligence Artificielle, IA pour les intimes, nous serons amenés à réaliser une série de projet permettant, comme chaque année,
l'application de la matière vue. Nous commençons fort avec ce premier projet de "Recherche" ayant pour but d'appliquer les algorithmes de recherche de plus
court chemin, à savoir \emph{BFS}, \emph{DFS} et finalement, \emph{A*} à l'aide de diverses structures de données comme le \emph{Stack} ou la \emph{PriorityQueue}.

Afin d'appliquer ces connaissances, il nous faut bien entendu un cadre. Celui-ci nous est proposé directement sous forme d'une librairie nommées sobrement
"\emph{Laser Learning Environment}", LLE pour les personnes friandes de termes succints. Elle nous permet d'avoir une grille en deux dimensions peuplée
d'éléments divers comme des agents, qui font le déplacement et des objectifs, c'est-à-dire un état à atteindre pour résoudre le problème donné. Ces
objectifs se manifestent sous forme de gemmes à collecter, de coins à visiter et de sorties à trouver (chaque objectif consiste en un problème à part
entière).

\section{Mode opératoire}
% création des cartes + image et heuristique pour GemProblem et CornerProblem
\subsection{Création des cartes}
Lors de la création de nos cartes personnalisées, notre but principal était de créer des environnements permettant des scénario d'exécution des algorithmes
uniques et distincts les un des autres. Ceci nous permet de bien nous rendre compte du comportement de chaque algorihme de recherche de chemin dans chaque cas
de manière claire.

\begin{itemize}
  \item \emph{EASY MAP}: Une carte totalement vide avec une gemme pour se rendre compte du cas le plus direct.
  \item \emph{ONE PATH MAP}: Cette carte ne possède qu'un seul chemin valide, permettant de se rendre compte de
  la robustesse des algorithmes, de leur temps d'exécution.
  \item \emph{MANY GEMS MAP}: Ici nous testons tous les paramètres sur une carte ouverte.
  \item \emph{COMPLEX MAP}: Pareil que pour la carte précédente mais avec plus d'obstacles pour mieux se rendre
  compte de certains aller-retours inutiles (en particulier dans le cas du DFS).
  \item \emph{IMPOSSIBLE MAP}: Cette dernière carte permet de vérifier la justesse des algorithmes de recherche
  dans le cas où il n'existe pas de chemin valide (voir s'il n'existe pas d'abberation).
\end{itemize}

\subsection{Heuristiques}


\section{Comparaison des algorithmes sur le GemProblem}
\subsection{Longueur du chemin}
\subsection{Durée moyenne d'exécution}
\subsection{Nombre de nœuds visités}
\section{Utilisation de L'IA}
\section{Conclusion}

\end{document}
